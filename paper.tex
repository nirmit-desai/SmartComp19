\documentclass[conference]{IEEEtran}
\IEEEoverridecommandlockouts
% The preceding line is only needed to identify funding in the first footnote. If that is unneeded, please comment it out.
\usepackage{cite}
\usepackage{amsmath,amssymb,amsfonts}
\usepackage{array,multirow,graphicx}
\usepackage{float}
\usepackage{algorithmic}
\usepackage{textcomp}
\usepackage{xcolor}
\def\BibTeX{{\rm B\kern-.05em{\sc i\kern-.025em b}\kern-.08em
    T\kern-.1667em\lower.7ex\hbox{E}\kern-.125emX}}

\newcommand{\taskprefix}{CH}
\newcommand{\tasksize}{\normalsize}
\newenvironment{taskenv}[1]{\begin{list}{{\tasksize\sc \theenumi.}}{\usecounter{enumi}
      \settowidth{\labelwidth}{{\tasksize\sc \taskprefix#1-99}}
      \setlength{\leftmargin}{\labelwidth}
      %\addtolength{\leftmargin}{2.0\labelsep}
}}{\end{list}}
\newcounter{task}
\setcounter{task}{-1}
\newcommand{\btask}{\begin{taskenv}{\taskprefix}\setcounter{enumi}{\value{task}}\renewcommand{\theenumi}{\taskprefix$_{\arabic{enumi}}$}}
\newcommand{\etask}{\setcounter{task}{\value{enumi}}\renewcommand{\theenumi}{\arabic{enumi}.}\end{taskenv}}

\begin{document}

\title{Large-scale hybrid ad hoc network for mobile platforms: Challenges and Experiences}

\author{\IEEEauthorblockN{Nirmit Desai, Wendy Chong, Heather Achilles, Shahrokh Daijavad}
\IEEEauthorblockA{\textit{IBM T. J. Watson Research Center} \\
%\textit{}\\
Yorktown Heights, NY, USA \\
\{nirmit.desai, wendych, hachilles, shahrokh\}@us.ibm.com}
\and
%\IEEEauthorblockN{Wendy Chong}
%\IEEEauthorblockA{\textit{IBM T. J. Watson Research Center} \\
%\textit{name of organization (of Aff.)}\\
%Yorktown Heights, NY, USA \\
%wendych@us.ibm.com}
%\and
%\IEEEauthorblockN{Heather Achilles}
%\IEEEauthorblockA{\textit{IBM T. J. Watson Research Center} \\
%\textit{name of organization (of Af}\\
%Yorktown Heights, NY, USA \\
%hachilles@us.ibm.com}
%\and
%\IEEEauthorblockN{Shahrokh Daijavad}
%\IEEEauthorblockA{\textit{IBM T. J. Watson Research Center} \\
%\textit{Penn}\\
%Yorktown Heights, NY, USA \\
%shahrokh@us.ibm.com}
\and
\IEEEauthorblockN{Thomas La Porta}
\IEEEauthorblockA{\textit{Dept. of Computer Science and Engineering} \\
\textit{Penn State University}\\
University Park, PA, USA \\
tlp@cse.psu.edu}
%\and
%\IEEEauthorblockN{5\textsuperscript{th} Given Name Surname}
%\IEEEauthorblockA{\textit{dept. name of organization (of Aff.)} \\
%\textit{name of organization (of Aff.)}\\
%City, Country \\
%email address}
%\and
%\IEEEauthorblockN{6\textsuperscript{th} Given Name Surname}
%\IEEEauthorblockA{\textit{dept. name of organization (of Aff.)} \\
%\textit{name of organization (of Aff.)}\\
%City, Country \\
%email address}
}

\maketitle

\begin{abstract}
Peer-to-peer (p2p) networks and Mobile ad hoc networks (MANET) have
been widely studied. However, a real-world deployment for the masses
has remained elusive. Ever-increasing density of mobile devices,
especially in urban areas, has given rise to new applications of p2p
communication. However, the modern smartphone platforms have limited
support for such communications. Further, the issues of battery life,
range, and trust remain unaddressed. A key question then is, what
kinds of applications can the modern mobile platforms support and what
challenges remain? This paper identifies a class of applications and
presents a novel center-to-peer-to-peer (c2p2p) architecture called
Mesh Network Alerts (MNA) to support them.  We describe our
experiences in deploying MNA as a real-world system to millions of
users for relaying severe weather information along with the
challenges faced, and the approaches for addressing them.
\end{abstract}

\begin{IEEEkeywords}
peer-to-peer systems, mobile ad hoc network, delay-tolerant network
\end{IEEEkeywords}

\section{Introduction}
Mobile devices with programmable platforms such as android and iOS
have steadily grown over the last decade, surpassing the 2 billion
mark \footnote{https://www.statista.com/statistics/330695/number-of-smartphone-users-worldwide/}. MANETs
have been widely studied given their decentralized nature and
potential for new applications
\cite{loo-manet-2011,perkins-ad-hoc-2001}. Most of the prior work on
p2p networks has focused on analytical and simulation-based study of
MANET behavior
\cite{zhang-topology-manet-2015,marti-misbehavior-manet-2000,mauve-pos-routing-manet-2001}. Given
the outstanding practical challenges in deployoing a large number of
physical nodes, real-world implementations have been limited and have
not reached mass scale \cite{kiess-manet-impl-2007}. However, with the
growth of smartphones, large-scale real-world implementations may
become feasible. This paper describes a real-world implementation of a
p2p delay-tolerant network, called Mesh Network Alerts (MNA), for
relaying severe weather information to millions of mobile device users
as part of the Weather Channel
app\footnote{https://weather.com/apps/ibm/meshnetworkalerts} on both
android and iOS platforms. 

Before describing MNA, it is critical to identify applications that
need p2p communication, given pervasive Internet connectivity. Doing
so enables us to define key characteristics of such applications and
focus on the challenges in meeting them.  This paper focuses on two
separate classes of applications: communication in
(a) disaster-affected or remote areas and (b) congested networks in
densely populated areas, e.g., sports arenas.  The following are the
key characteristics in these scenarios:

%
\btask
%
\item\label{c:0} No communication infrastructure such as WiFi
  access points to fall back on
\item\label{c:1} Network nodes are mobile, pattern of mobility is not
  predictable
\item\label{c:2} New information may arrive at any time
\item\label{c:3} Trustworthy information is scarce, misinformation and
  rumours are common place
\item\label{c:4}  Small payloads suffice in many cases and information
  retains value for a few minutes
\item\label{c:5}  Device battery is a scarce resource, power supply
  for recharging may not be available
\item\label{c:6}  Devices are owned by citizens, deployment of
  special-purpose devices is cost prohibitive
%
\etask
%

The above needs are well-recognized in the industry as well as
academia with several ambitious attempts to address them, e.g., Google
Loon project\footnote{https://loon.co} and Facebook
Aquilla\footnote{https://en.wikipedia.org/wiki/Facebook\_Aquila},
though with limited impact. Leveraging user mobile devices as peer
nodes for a large-scale deployment has been another theme in the prior
works, e.g., the Serval project
\cite{gardner-stephen-serval-2011}. Serval mesh enables p2p
communication over on-device WiFi radio, but requires root access to
the device via jailbreaking. Although significant leassons have been
learned through these attempts, a mass-scale p2p network for such
applications remains elusive.

A vast majority of the literature has focused on a traditional model
of stateful, fully decentralized, reliable networking. Specifically,
the nodes maintain connectivity with peers and routing is optimized
with techniques based on link state or distance vectors
\cite{clausen-olsr-2003,perkins-aodv-2003} focusing on optimizing the
network utilization.

Given the application characteristics above, this paper identifies
main practical challenges associated with modern device platforms and
finds novel ways to overcome them.  This leads to MNA --- a new
paradigm in p2p networking that employs a hybrid, delay-tolerant, and
zero-routing overhead architecture.  Unlike previous MANET
architectures, MNA is a hybrid of a centralized and a decentralized
architecture, called \emph{center-to-peer-to-peer (c2p2p)}. A central service
is leveraged as the trusted source of information while the
information is propagated in a decentralized fashion by peers. Such an
architecture allows MNA to bypass the key issue of trust deficit in
open decentralized systems and scale while retaining the ability of
infrastructure-less communication in many application scenarios.

MNA is implemented on both Android and iOS as an SDK and integrated
with the Weather Channel mobile apps. With extensive experiments and
experience of deploying to almost $10$ million users, MNA represents a
way forward for large-scale p2p networks.

This paper makes the following key contributions:
\begin{itemize}
\item Identification of a class of applications for p2p and their key
  characteristics
\item A deeper investigation of the practical challenges in supporting
  the above class of applications
\item A novel c2p2p architecture implementation using multiple radio
  channels for modern mobile platforms (Android and iOS)
\item Experimental evaluation and deployment statistics
\end{itemize}

In the following, Section~\ref{sec:challenges} describes the practical
challenges. Section~\ref{sec:architecture} outlines the architectural
details of MNA along with platform-specific implementation issues for
Android and iOS in addressing the challenges. Experimental evaluation
and deployment statistics are presented in Section~\ref{sec:eval}.  A
deeper look at the literature and contrast to MNA is summarized in
Section~\ref{sec:related} with conclusions in
Section~\ref{sec:conclude}.
%
\section{Challenges}
\label{sec:challenges}
%
We present the main challenges for modern mobile device platforms in
supporting the classes of applications described above. These have been
uncovered via extensive experiments and in some cases include direct
feedback from the developers of Android and iOS.
%
\subsection{Operating system constraints}
\label{ch:os}
%
Due to \ref{c:2}, even when a user is not interacting with an app, or
worse yet, when the device is not being used at all, the devices must
continue to discover peer devices to receive and formward
information. Although modern mobile operating systems such as Android
and iOS offer APIs to discover and advertise information to peer
devices over WiFi and Bluetooth interfaces, peer-to-peer connections
do not work reliably when the same APIs are accessed while the app is
in the background. Further, on Android, each WiFi p2p connection must
be explicitly approved by the device user. Prior works widely document
these challenges and take the approach of having special access on the
devices, e.g., jail-breaking or rooting
\cite{gardner-stephen-serval-2011}. Clearly, such an approach does not
scale to mass adoption.\\
%
%
\subsection{Power constraints}
\label{ch:power}
%
Since devices may be offline when new weather information arrives, MNA
on each peer must remain active at all times to be able to discover
new information as soon as it arrives. Further, as there is no back up
infrastructure (\ref{c:0}) and a set of peers in range can change at
any time (\ref{c:1}), each peer is responsible for constantly
forwarding available information to other peers via
advertisements. However, due to \ref{c:5}, the MNA activity must keep
the device battery consumption to a minimum. This is a challenge
because advertising and discovery are power-hungry operations over the
radio channels.
%
\subsection{Testing p2p networks}
%
Given the heterogeneity of devices owned by users and operating system
distributions (\ref{c:6}), it is challenging to test whether or not
MNA functions as expected on a single device. Further, running test
scenarios on a p2p network at large-scale is non-trivial given that a
large number of devices need to take specific coordinated action
followed by coordinated observations to determine whether a test
passes or fails. Further, since range and mobility affect p2p
communications and they are unpredictable (\ref{c:1}), it is important
to run test cases under various mobility patterns across all nodes in
the network.  Emulated mobility frameworks such as CORE
\cite{arenholz-core-2008} and EMANE \cite{emane} fall short as the
connection latencies and wireless transmission are specific to device
model and radio.  This is not a challenge in traditional mobile
application development as the application functionality is confined
within a single connected device.
%
\subsection{Trust in information}
%
As user devices with MNA advertise on unsecure wireless protocols, it
may be possible for a malicious attacker to listen for such
advertisements and reverse-engineer the message formats and protocols
used. Then, the attackers may generate fake messages and advertise
them, e.g., a fake tornado alert. Given a lack of trusted information
in such scenarios (\ref{c:3}), misinformation campaigns can have
disastrous consequences.  In open decentralized systems, such false
messages cannot be distinguished from the real ones, and MNA will end
up propagating them to as many devices as possible, ``poisoning'' the
network. In general, veracity of such information cannot be
independently verified in open decentralized systems and previous works
on peer-to-peer networks do not address this challenge.
%
\section{C2P2P System Architecture}
\label{sec:architecture}
%
\begin{figure}[htbp]
\centerline{\includegraphics[width=\columnwidth]{figs/arch}}
\caption{c2p2p system architecture}
\label{fig:arch}
\end{figure}

To address (or bypass) the challenges identified above, this paper
proposes c2p2p -- a hybrid architecture that leverages a central
service as the sole source of trusted information as depicted in
Figure~\ref{fig:arch}. An application-specific central service, e.g.,
weather.com backend, originates new information as a push message and
assigns it a unique identifier. Next, the central service produces a
digital signature using its private key and appends it to the message
payload. Finally, header parameters specifying TTL (duration after
which the message expires), peer identifier of the central service,
destination peer identifier list (for unicast) or a special identifier
(for broadcast), and the time of origin are added to the message and
the message is sent to a push notification service for distribution.

Mobile applications integrate MNA as an SDK including the public key
of the corresponding central service. Peers having Internet
connectivity receive the push messages and verify the digital
signature to protect against spread of misinformation.  If the message
has not expired and is destined to other peers, then the receiving
peers start forwarding the message to other peers. Depending on the
protocol being used, such forwarding happens as a unicast or
broadcast.

Since new information can arrive at any time or new peers may come
into proximity, and message transmissions are not reliable, all peers
store unexpired and verified messages and continue to forward them to
proximal peers repeatedly. This implies that peers may receive the
same message more than once but since the messages are globally
unique, duplicates can be identified and ignored.  A peer sending a
message adds its own peer identifier to a list of forwarders appended
to the message header.  A receiving peer thus knows all the peers that
already have the message and stops repeating the message to those
peers, controlling the flooding.  As there are no control messages or
any other overhead for routing, we call this \emph{zero-overhead}
routing.

The following describes how the c2p2p architecture addresses the
challenges identified above along with known limitations. Further, to
realize this architecture on mobile devices, we evaluated a variety of
protocols on both Android and iOS. As described later, WiFi DNS on
Android, Bluetooth Nearby on Android, and Blutooth low energy (BTLE)
on both Android and iOS are the only protocols that were found
effective. Actual techniques for discovery, advertisement, and
connection vary across these protocols and are described next.
%
\subsection{Addressing the challenges}
\label{sec:address}
%
\noindent\textbf{Operating system constraints} MNA addresses these
challenges via innovative techniques, without resorting to hacks that
may violate user security or App-store guidelines. DNS service
discovery is a widely supported protocol for all platforms
\cite{cheshire-dns-sd-2013}. 

On Android, WiFi DNS improvises on this standard such that the
exchange of information happens without making network connections at
all. This is achieved by splitting the message payloads into small
enough chunks and stuffing them into service advertisements as txt
records and broadcast over multiple advertisements in a quick
succession. Also, we employ foreground services to keep the discovery
and advertisements active indefinitely.

On iOS, the only protocol that stays reliably active in background
without violating iOS developer guidelines is BTLE. Unlike Android,
iOS applications do not need to request discovery and advertisements
contunually. Once a BTLE Peripheral (role that has information) is
advertised and a Central (role that receives information) requests
discovery, the app is allowed to go to sleep. When a matching central
or a peripheral device is found, iOS wakes up the application app to
handle such an event, even when the application is in the
background. MNA builds on this and leverages acynchronous dispatch
queues to turn each peer into a Central as well as a Peripheral
simultaneously. This architecture allows bidirectional data transfers
without concurrency issues.

\noindent\textbf{Power constraints} Due to indefinite forground
services and continuous discovery and advertisement, this is primarily
an issue on Android. Our approach here is two-pronged. Firstly, via
extensive experiments on a large number of devices, we fine-tune the
algorithms governing the intervals at which discovery and
advertisements occur. In a nutshell, receiving new information during
a period causes MNA to be more aggressive in discovery and
advertisement. Similarly, lack of new information for a period makes
the device less aggressive. Secondly, we allow ``wake up'' messages to
be broadcast in the network ahead of an anticipated severe weather
event. When devices receive such messages, they schedule themselves to
remain aggressive during the specified window of time. Outside of this
window, the device can afford to have long sleep cycles and conserve
power. With these techniques, our testing shows less than 2\% battery
consumption per hour on most device models.

\noindent\textbf{Testing p2p networks}
We developed test automation tools and processes such that multiple
devices can be controlled from a single test station and follow
prescribed steps to generate, send, and receives messages to play out
a test scenario. Finally, the framework allows automated analysis of
the observations to determine the test result. This capability was
instrumental in uncovering bugs at a fast pace to meet the aggressive
timeline and avoid the cost of acquisition. Further, the automation
framework is general and can be expressly applied to test other apps
in this fashion.


\noindent\textbf{Trust in information} In our approach, since the
origin of weather information is the Weather Channel service, digital
signatures can be attached to all weather alerts broadcast from the
service. The mobile application is distributed with the corresponding
public key so that digital signatures from the service can be
verified. If a message fails such a verification, it is discarded and
not forwarded any further.

\begin{figure}[htbp]
\centerline{\includegraphics[width=\columnwidth]{figs/android_ios}}
\caption{Interoperability matrix between Android and iOS devices over BTLE}
\label{fig:android_ios}
\end{figure}


\begin{table}[H]
\caption{Android protocols and range results}
\centering
\begin{tabular}{|c|l|r|r|r|c|l|}
\hline
& \multicolumn{1}{c|}{Protocol} & \multicolumn{1}{c|}{Throughput} & \multicolumn{1}{c|}{Battery} & \multicolumn{1}{c|}{Range} & \multicolumn{1}{c|}{iOS} & \multicolumn{1}{c|}{Issues}\\
&                               & \multicolumn{1}{c|}{Avg. kbps}  & \multicolumn{1}{c|}{per hour} & \multicolumn{1}{c|}{LoS ft} &                  &                            \\
\hline
\parbox[t]{2mm}{\multirow{3}{*}{\rotatebox[origin=c]{90}{WiFi}}} & \textbf{DNS} & 0.2 & 2\% & 600 & No & \\
& WiDi & 2000 & 3\% & 600 & No & Needs DNS\\
& Hotspot & 2000 & 5\% & 600 & No & Permission\\
\hline
\parbox[t]{2mm}{\multirow{3}{*}{\rotatebox[origin=c]{90}{BT}}} & Classic & 50 & 2\% & 600 & No & Unstable\\
& \textbf{Nearby} & 50 & 2\% & 600 & No & \\
& \textbf{BTLE} & 50 & 2\% & 600 & Yes & \\
\hline
\end{tabular}
\end{table}


\begin{table}[H]
\caption{iOS protocols and range results}
\centering
\begin{tabular}{|c|l|r|r|r|c|l|}
\hline
& \multicolumn{1}{c|}{Protocol} & \multicolumn{1}{c|}{Throughput} & \multicolumn{1}{c|}{Battery} & \multicolumn{1}{c|}{Range} & \multicolumn{1}{c|}{Android} & \multicolumn{1}{c|}{Issues}\\
&                               & \multicolumn{1}{c|}{Avg. kbps}  & \multicolumn{1}{c|}{per hour} & \multicolumn{1}{c|}{LoS ft} &                  &                            \\
\hline
\parbox[t]{2mm}{\multirow{3}{*}{\rotatebox[origin=c]{90}{WiFi}}} & Bonjour & 0.2 & 5\% & 200 & No & Battery\\
& MPC & 2000 & 5\% & 200 & No & Battery\\
& WiDi & 2000 & 5\% & 200 & No & Battery\\
\hline
& \textbf{BTLE} & 50 & 2\% & 800 & Yes & \\
\hline
\end{tabular}
\end{table}


%
\subsection{Automated test framework}
\label{sec:test}
%

\begin{figure}[htbp]
\centerline{\includegraphics[width=\columnwidth]{figs/test_arch}}
\caption{Automated testing system for MNA}
\label{fig:test_arch}
\end{figure}

%
\section{Experimental Results}
\label{sec:eval}
%

%

%

%
\subsection{Pilot test results}
\label{sec:romania}
%
\begin{figure}[htbp]
\centerline{\includegraphics[width=\columnwidth]{figs/romania_latency}}
\caption{Latency results from a 25-user pilot}
\label{fig:romania_lat}
\end{figure}

\begin{figure}[htbp]
\centerline{\includegraphics[width=\columnwidth]{figs/romania_activity}}
\caption{MNA activity during the pilot}
\label{fig:romania_act}
\end{figure}


About 13\% of the messages were received at least once (depends on
density of users in the building). Users actively used the devices for
~5\% of the time p2p NSD achieved lower end-to-end latencies than BT
Classic, perhaps due to its broadcast model and wide support across
devices. Packet sizes did not affect single-hop latencies on p2p DNS,
perhaps because most of the messages fit within single advert. Packet
sizes did not affect single-hop latencies on BT Classic, perhaps
because Bluetooth has a much higher bandwidth 2 out of 25 devices have
their clocks set wrong (confusing latency results). More than 10\% of
the messages were received in the last 5 min of message expiry,
implying messages were discarded due to expirty, even when they did
not reach all devices

%
\subsection{Automated test results}
\label{sec:automated}
%
\begin{figure}[htbp]
\centerline{\includegraphics[width=\columnwidth]{figs/variety_e2e_latency}}
\caption{Comparison of WiFi and Bluetooth interfaces in terms of
  end-to-end latency}
\label{fig:variety_e2e}
\end{figure}

\begin{figure}[htbp]
\centerline{\includegraphics[width=\columnwidth]{figs/variety_hops}}
\caption{Comparison of WiFi and Bluetooth interfaces in terms of
  network topology}
\label{fig:variety_hops}
\end{figure}


\begin{figure}[htbp]
\centerline{\includegraphics[width=\columnwidth]{figs/e2e_latency}}
\caption{Comparison of WiFi and Bluetooth Nearby interfaces in terms
  of end-to-end latency}
\label{fig:e2e}
\end{figure}

\begin{figure}[htbp]
\centerline{\includegraphics[width=\columnwidth]{figs/duplicates}}
\caption{Duplicity due to flooding}
\label{fig:dup}
\end{figure}

\begin{figure}[htbp]
\centerline{\includegraphics[width=\columnwidth]{figs/hops}}
\caption{Average hop count}
\label{fig:hop}
\end{figure}

%
\section{Related Work}
\label{sec:related}
%
Apart from the state-of-the-art cited earlier, there have been other
notable attempts at practical large-scale deployments and we describe
them here. Before mid-2000’s, most of the research on MANETs was based
on Department of Defense requirements, until commodity multi-hop ad
hoc networks began to be considered \cite{bruno-mesh-2005}. However,
it took another decade before wireless mesh networking was used
commercially to enable smartphones to connect via Bluetooth and WiFi
in a popular application called FireChat \cite{firechat}. The success
of FireChat, partially due to the news coverage of its use in
political situations in which governments restricted access to the
Internet, has led to many alternatives in the past few years. An
up-to-date list of such applications is available
here \footnote{https://alternativeto.net/software/firechat-by-open-garden/}
\section{Conclusions and Future Work}
\label{sec:conclude}

\section*{Acknowledgment}
This research was sponsored by the U.S. Army Research Laboratory and
the U.K. Ministry of Defence under Agree- ment Number
W911NF-16-3-0001. The views and conclusions contained in this document
are those of the authors and should not be interpreted as representing
the official policies, either expressed or implied, of the U.S. Army
Research Laboratory, the U.S. Government, the U.K. Ministry of Defence
or the U.K. Government. The U.S. and U.K. Governments are au- thorized
to reproduce and distribute reprints for Government purposes
notwithstanding any copyright notation hereon.
\bibliographystyle{abbrv} \bibliography{smartcomp19}

\end{document}
